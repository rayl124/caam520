% Document setup
\documentclass[12pt,t]{beamer}
\usetheme[outer/progressbar=none]{metropolis}
% Algorithm environment and layout
\usepackage{algorithmicx}
\usepackage{algpseudocode}
\algloopdefx{Return}{\textbf{return} }
\usepackage{amsmath}
\usepackage{amssymb}
\usepackage{mathtools}
\usepackage[T1]{fontenc}

\usepackage{xparse}
\newbox\FBox
\NewDocumentCommand\Highlight{O{black}O{white}mO{0.5pt}O{0pt}O{0pt}}{%
    \setlength\fboxsep{#4}\sbox\FBox{\fcolorbox{#1}{#2}{#3\rule[-#5]{0pt}{#6}}}\usebox\FBox}

\usepackage{float}
\usepackage{graphicx}
\usepackage[outdir=./]{epstopdf}
\usepackage{color}
\newcommand{\red}[1]{{\color{red}#1}}
\usepackage{caption}
\definecolor{RiceBlue}{HTML}{004080}
\definecolor{BackgroundColor}{rgb}{1.0,1.0,1.0}
\setbeamercolor{frametitle}{bg=RiceBlue}
\setbeamercolor{background canvas}{bg=BackgroundColor}
\setbeamercolor{progress bar}{fg=RiceBlue}
\setbeamertemplate{caption}[default]

\usepackage{multirow}
\usepackage{tabularx}

\usepackage{xcolor}

\usepackage{listings}
\lstset{basicstyle=\footnotesize,showstringspaces=false}

\usepackage{adjustbox}

\usepackage[absolute,overlay]{textpos}

%% Overwrite font settings to make text and math font consistent.
\usepackage[sfdefault,lining]{FiraSans}
\usepackage[slantedGreek]{newtxsf}
\renewcommand*\partial{\textsf{\reflectbox{6}}}
\let\emph\relax % there's no \RedeclareTextFontCommand
\DeclareTextFontCommand{\emph}{\bfseries\em}

\renewcommand*{\vec}[1]{{\boldsymbol{#1}}}

\newcommand{\conclude}[1]{%
  \begin{itemize}
    \item[$\rightarrow$]#1
  \end{itemize}
}
\newcommand{\codeline}[2][]{%
  \begin{lstlisting}[language=c++,#1]^^J
    #2^^J
  \end{lstlisting}
}
\newcommand{\codefile}[2][]{\lstinputlisting[language=c++,frame=single,breaklines=true,#1]{#2}}
\lstnewenvironment{code}{\lstset{language=c++,frame=single}}{}
\newcommand{\cmd}[1]{\begin{center}\texttt{#1}\end{center}}


\begin{document}
  % Title page
  \title{Tools of the Trade -- Part I:\\Compiler, Linker, Build System}
  \subtitle{Computational Science II (CAAM 520)}
  \author{Christopher Thiele}
  \date{Rice University, Spring 2020}

  \setbeamertemplate{footline}{}
  \begin{frame}
    \titlepage
  \end{frame}

  \setbeamertemplate{footline}{
    \usebeamercolor[fg]{page number in head}%
    \usebeamerfont{page number in head}%
    \hspace*{\fill}\footnotesize-\insertframenumber-\hspace*{\fill}
    \vspace*{0.1in}
  }

  \begin{frame}[fragile]
    \frametitle{Overview}

    Now that we can write C code, how do we transform it into a program?
    \begin{itemize}
      \item Compiler: Translates C code to machine code.
      \item Linker: Transforms machine code to a binary, e.g., for Linux.
      \item Build system: Manages the compiler and linker (among other things).
    \end{itemize}
  \end{frame}

  \begin{frame}[fragile]
    \frametitle{Compiler}

    While there are countless C compilers available, we will focus on GCC.
    \begin{itemize}
      \item Can be considered the default on GNU/Linux.
      \item Most other compilers for Linux/Unix are largely compatible (e.g., Clang, Intel Compiler).
    \end{itemize}
  \end{frame}

  \begin{frame}[fragile]
    \frametitle{Compiler}

    To compile a single source file \texttt{test.c} in to an \emph{executable} or \emph{binary} called \texttt{test}, we simply run
    \cmd{gcc -o test test.c}
  \end{frame}

  \begin{frame}[fragile]
    \frametitle{Compiler}

    Use the compiler to your advantage!

    The compiler can help you with a lot of warnings if you write code that is not robust and potentially faulty:
    \cmd{gcc -Wall -Wextra -Wpedantic -o test test.c}
  \end{frame}

  \begin{frame}[fragile]
    \frametitle{Compiler}
    So far we compiled and linked in one step! Let's look at both steps separately.

    If we only want to compile, i.e., translate C code into machine code, we run
    \cmd{gcc -c -o test.o test.c}

    \conclude{Produces an \emph{object file}.}
  \end{frame}

  \begin{frame}[fragile]
    \frametitle{Linker}

    To create an executable from one or many object files, we run
    \cmd{gcc -o test test1.o test2.o}

    \conclude{Linking}
  \end{frame}

  \begin{frame}[fragile]
    \frametitle{Linker}

    The linker also includes \emph{libraries} and third-party code.
    \cmd{gcc -o test test.o \emph{-lm}}

    \conclude{The flag \texttt{-lm} tells the linker to link (l) against the math library (m), which contains functions like \texttt{exp()}.}
  \end{frame}

  \begin{frame}[fragile]
    \frametitle{Linker}

    Why would we want to separate compilation and linking?

    Compiling everything at once might be undesirable or impossible:
    \begin{itemize}
      \item If we have 100 source files and there is an error in one of them, the compilation of all 100 has to be repeated.
      \item If we compile files separately, we can do so in parallel.
    \end{itemize}
  \end{frame}

  \begin{frame}[fragile]
    \frametitle{Build system}

    Who has 100 source files?
    \conclude{Linux kernel has more than 10,000,000 lines of code.}

    Does Linus Torvalds type
    \cmd{gcc -c -o file.o file.c}
    hundreds of times?

    \conclude{No, he uses a build system.}
  \end{frame}

  \begin{frame}[fragile]
    \frametitle{Build system}

    We will use GNU Make as a build system.

    It allows us to compile and link an application accorting to a recipe, a so-called \emph{Makefile}:

    \begin{lstlisting}[frame=single]
# Simplest structure of a target in a Makefile:
target: dependencies
  things_to_do
    \end{lstlisting}
  \end{frame}

  \begin{frame}[fragile]
    \frametitle{Build system}

    A Makefile to compile a program from two source files:

    \begin{lstlisting}[frame=single]
my_application: file1.c file2.c
# Simplest structure of a target in a Makefile:
  # Indent with *tabs*, not with spaces!
  gcc -o my_application file1.c file2.c
    \end{lstlisting}
  \end{frame}

  \begin{frame}[fragile]
    \frametitle{Build system}

    Makefiles can automate much more complex build processes:

    \begin{lstlisting}[frame=single]
SRC := $(wildcard *.c)
OBJ := $(patsubst %.c, %.o, $(SRC))

my_application: $(OBJ)
  gcc -o $@ $^

%.o: %.c
  gcc -c -o $@ $<
    \end{lstlisting}

    \conclude{Unfortunately, the syntax is horrible.}
  \end{frame}

  \begin{frame}[fragile]
    \frametitle{Build system}

    Why do we need Makefiles in class?

    To ensure that your code can be built without knowing the details!

    \begin{lstlisting}[frame=single]
# Student 1
hw1: file.c
  gcc -o hw1 file.c
    \end{lstlisting}

    \begin{lstlisting}[frame=single]
# Student 2
hw1: file1.cc file2.cc
  g++ -std=c++14 -o hw1 file1.cc file2.cc -lm
    \end{lstlisting}
  \end{frame}

  \begin{frame}[fragile]
    \frametitle{Build system}

    Note that (GNU) Make has its limitations:
    \begin{itemize}
      \item Horrible syntax
      \item First appeared in 1976
      \item Targets Unix/Linux
    \end{itemize}

    \conclude{For more complex tasks, consider using a meta build system like CMake.}
  \end{frame}
\end{document}
