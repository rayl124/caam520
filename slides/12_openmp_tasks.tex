% Document setup
\documentclass[12pt,t]{beamer}
\usetheme[outer/progressbar=none]{metropolis}
% Algorithm environment and layout
\usepackage{algorithmicx}
\usepackage{algpseudocode}
\algloopdefx{Return}{\textbf{return} }
\usepackage{amsmath}
\usepackage{amssymb}
\usepackage{mathtools}
\usepackage[T1]{fontenc}

\usepackage{xparse}
\newbox\FBox
\NewDocumentCommand\Highlight{O{black}O{white}mO{0.5pt}O{0pt}O{0pt}}{%
    \setlength\fboxsep{#4}\sbox\FBox{\fcolorbox{#1}{#2}{#3\rule[-#5]{0pt}{#6}}}\usebox\FBox}

\usepackage{float}
\usepackage{graphicx}
\usepackage[outdir=./]{epstopdf}
\usepackage{color}
\newcommand{\red}[1]{{\color{red}#1}}
\usepackage{caption}
\definecolor{RiceBlue}{HTML}{004080}
\definecolor{BackgroundColor}{rgb}{1.0,1.0,1.0}
\setbeamercolor{frametitle}{bg=RiceBlue}
\setbeamercolor{background canvas}{bg=BackgroundColor}
\setbeamercolor{progress bar}{fg=RiceBlue}
\setbeamertemplate{caption}[default]

\usepackage{multirow}
\usepackage{tabularx}

\usepackage{xcolor}

\usepackage{listings}
\lstset{basicstyle=\footnotesize,showstringspaces=false}

\usepackage{adjustbox}

\usepackage[absolute,overlay]{textpos}

% Footnotes without a number
\newcommand\blfootnote[1]{%
  \begingroup
  \renewcommand\thefootnote{}\footnote{\tiny #1}%
  \addtocounter{footnote}{-1}%
  \endgroup
}

%% Overwrite font settings to make text and math font consistent.
\usepackage[sfdefault,lining]{FiraSans}
\usepackage[slantedGreek]{newtxsf}
\renewcommand*\partial{\textsf{\reflectbox{6}}}
\let\emph\relax % there's no \RedeclareTextFontCommand
\DeclareTextFontCommand{\emph}{\bfseries\em}

\renewcommand*{\vec}[1]{{\boldsymbol{#1}}}

\newcommand{\conclude}[1]{%
  \begin{itemize}
    \item[$\rightarrow$]#1
  \end{itemize}
}
\newcommand{\codeline}[2][]{%
  \begin{lstlisting}[language=c++,#1]^^J
    #2^^J
  \end{lstlisting}
}
\newcommand{\codefile}[2][]{\lstinputlisting[language=c++,frame=single,breaklines=true,#1]{#2}}
\lstnewenvironment{code}{\lstset{language=c++,frame=single}}{}
\newcommand{\cmd}[1]{\begin{center}\texttt{#1}\end{center}}


\begin{document}
  % Title page
  \title{OpenMP Tasks}
  \subtitle{Computational Science II (CAAM 520)}
  \author{Christopher Thiele}
  \date{Rice University, Spring 2020}

  \setbeamertemplate{footline}{}
  \begin{frame}
    \titlepage
  \end{frame}

  \setbeamertemplate{footline}{
    \usebeamercolor[fg]{page number in head}%
    \usebeamerfont{page number in head}%
    \hspace*{\fill}\footnotesize-\insertframenumber-\hspace*{\fill}
    \vspace*{0.1in}
  }

  \begin{frame}[fragile]
    \frametitle{Motivation}

    So far all of our OpenMP code was loop based.

    How can we implement more general multi-threaded algorithms?
  \end{frame}

  \begin{frame}[fragile]
    \frametitle{Tasks}

    OpenMP tasks allow us to offload any "chunk" of work to a thread.
    \begin{code}
#pragma omp parallel
{
  #pragma omp single
  {
    #pragma omp task
    foo();

    #pragma omp task
    bar();
  }
}
    \end{code}
  \end{frame}

  \begin{frame}[fragile]
    \frametitle{Tasks}

    In this example \texttt{foo()} and \texttt{bar()} will be executed as tasks.

    The OpenMP runtime environment schedules the tasks:
    \begin{itemize}
      \item Tasks can be executed by any thread.
      \item Tasks can be executed in any order.
    \end{itemize}
    \conclude{Why do we need a \texttt{single} directive inside the parallel region?}
  \end{frame}

  \begin{frame}[fragile]
    \frametitle{Data environment for tasks}

    A task views variables as
    \begin{itemize}
      \item \texttt{firstprivate}, if the variable was \texttt{private} to the thread that created the task.
      \item \texttt{shared}, if the variable was \texttt{shared} in the thread that created the task.
    \end{itemize}
  \end{frame}

  \begin{frame}[fragile]
    \frametitle{Data environment for tasks}

    \begin{code}
int i;

#pragma omp parallel
{
  int j;

  #pragma omp task
  {
    // Task has a "shared" view of i.
    // Task has a "firstprivate" copy of j.
  }
}
    \end{code}
  \end{frame}

  \begin{frame}[fragile]
    \frametitle{Data environment for tasks}

    The data environment can become complicated.

    To be safe, use defaults!

    \begin{code}
int i;
#pragma omp parallel
{
  int j;
  #pragma omp task shared(i) firstprivate(j) \
                   default(none)
  {
    // The default(none) clause requires that
    // any variable is declared as shared or
    // firstprivate *explicitly*!
  }
}
    \end{code}
  \end{frame}

  \begin{frame}[fragile]
    \frametitle{Tasks and barriers}

    An OpenMP barrier enforces the completion of \emph{all} incomplete tasks that were created in the current parallel environment.

    \begin{code}
#pragma omp parallel
{
  #pragma omp single
  for (int i = 0; i < 4; i++) {
    #pragma omp task
    do_work(i);
  }
  #pragma omp barrier

  // All tasks are complete at this point.
}
    \end{code}
  \end{frame}

  \begin{frame}[fragile]
    \frametitle{The \texttt{taskwait} directive}

    The \texttt{taskwait} directive enforces the completion of all \emph{child} tasks.

    It does not synchronize threads in any other way, i.e., it is weaker than a barrier.
  \end{frame}

  \begin{frame}[fragile]
    \frametitle{Tasks and dependencies}

    OpenMP allows us to express dependencies between tasks:
    \begin{code}
#pragma omp parallel
{
  #pragma omp single
  {
    #pragma omp task depend(out:x)
    x = foo();
    // First task must finish before
    // this task can run.
    #pragma omp task depend(in:x)
    bar(x);
  }
}
    \end{code}
  \end{frame}

  \begin{frame}[fragile]
    \frametitle{Tasks and dependencies}

    Dependencies can be expressed by adding \texttt{depend(type:list)} clauses to the \texttt{task} directive, where
    \begin{itemize}
      \item \texttt{type} can be \texttt{in}, \texttt{out}, \texttt{inout}, among other options, and
      \item \texttt{list} is a comma-separated list of variables.
    \end{itemize}
    \conclude{Instead of individual variables, we can also specify ranges within arrays in the format \texttt{my\_array[start:length]}.}
  \end{frame}
\end{document}
