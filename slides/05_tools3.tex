% Document setup
\documentclass[12pt,t]{beamer}
\usetheme[outer/progressbar=none]{metropolis}
% Algorithm environment and layout
\usepackage{algorithmicx}
\usepackage{algpseudocode}
\algloopdefx{Return}{\textbf{return} }
\usepackage{amsmath}
\usepackage{amssymb}
\usepackage{mathtools}
\usepackage[T1]{fontenc}

\usepackage{xparse}
\newbox\FBox
\NewDocumentCommand\Highlight{O{black}O{white}mO{0.5pt}O{0pt}O{0pt}}{%
    \setlength\fboxsep{#4}\sbox\FBox{\fcolorbox{#1}{#2}{#3\rule[-#5]{0pt}{#6}}}\usebox\FBox}

\usepackage{float}
\usepackage{graphicx}
\usepackage[outdir=./]{epstopdf}
\usepackage{color}
\newcommand{\red}[1]{{\color{red}#1}}
\usepackage{caption}
\definecolor{RiceBlue}{HTML}{004080}
\definecolor{BackgroundColor}{rgb}{1.0,1.0,1.0}
\setbeamercolor{frametitle}{bg=RiceBlue}
\setbeamercolor{background canvas}{bg=BackgroundColor}
\setbeamercolor{progress bar}{fg=RiceBlue}
\setbeamertemplate{caption}[default]

\usepackage{multirow}
\usepackage{tabularx}

\usepackage{xcolor}

\usepackage{listings}
\lstset{basicstyle=\footnotesize,showstringspaces=false}

\usepackage{adjustbox}

\usepackage[absolute,overlay]{textpos}

%% Overwrite font settings to make text and math font consistent.
\usepackage[sfdefault,lining]{FiraSans}
\usepackage[slantedGreek]{newtxsf}
\renewcommand*\partial{\textsf{\reflectbox{6}}}
\let\emph\relax % there's no \RedeclareTextFontCommand
\DeclareTextFontCommand{\emph}{\bfseries\em}

\renewcommand*{\vec}[1]{{\boldsymbol{#1}}}

\newcommand{\conclude}[1]{%
  \begin{itemize}
    \item[$\rightarrow$]#1
  \end{itemize}
}
\newcommand{\codeline}[2][]{%
  \begin{lstlisting}[language=c++,#1]^^J
    #2^^J
  \end{lstlisting}
}
\newcommand{\codefile}[2][]{\lstinputlisting[language=c++,frame=single,breaklines=true,#1]{#2}}
\lstnewenvironment{code}{\lstset{language=c++,frame=single}}{}
\newcommand{\cmd}[1]{\begin{center}\texttt{#1}\end{center}}


\begin{document}
  % Title page
  \title{Tools of the Trade -- Part III:\\Memory Debugger and Profiler}
  \subtitle{Computational Science II (CAAM 520)}
  \author{Christopher Thiele}
  \date{Rice University, Spring 2020}

  \setbeamertemplate{footline}{}
  \begin{frame}
    \titlepage
  \end{frame}

  \setbeamertemplate{footline}{
    \usebeamercolor[fg]{page number in head}%
    \usebeamerfont{page number in head}%
    \hspace*{\fill}\footnotesize-\insertframenumber-\hspace*{\fill}
    \vspace*{0.1in}
  }

  \begin{frame}[fragile]
    \frametitle{Memory debugging}

    Recall that improper memory management can cause leaks (in languages like C and C++):
    \begin{code}
void foo(const void *data, size_t size)
{
  void *copy = malloc(size);

  if (!data) {
    return; // Memory leak!
  }

  // ...

  free(copy);
}
    \end{code}
  \end{frame}

  \begin{frame}[fragile]
    \frametitle{Memory debugging}

    In the example, the memory leak is obvious.
    \conclude{How to find leaks in a large, complex code base?}

    We can use a memory debugger like Valgrind:
    \cmd{valgrind {-}-tool=memcheck ./myapp}

    Valgrind will point out definite and possible memory leaks.

    \emph{Note:} Running your code in Valgrid will slow down execution \emph{significantly}!
  \end{frame}

  \begin{frame}[fragile]
    \frametitle{Valgrind}

    Besides memory debugging, Valgrind has additional features:
    \begin{itemize}
      \item Identify (some types) of performance bottlenecks.
      \item Find errors in parallel applications.
    \end{itemize}

    These features will not be covered in class, but you may want to try them at some point.
  \end{frame}

  \begin{frame}[fragile]
    \frametitle{Profiling}

    Before we attempt to write parallel and performance-conscious code, we should understand how to assess the performance of an application.
    \begin{itemize}
      \item Where in the code does the application spend most time?
      \item What are the performance hotspots and bottlenecks?
      \item Where should we start optimizing?
    \end{itemize}
  \end{frame}

  \begin{frame}[fragile]
    \frametitle{Profiling}

    \emph{Profilers} can help us to collect this information.

    The most common approaches are:
    \begin{itemize}
      \item \emph{Instrumentation:} The compiler automatically adds timers, counters, etc. to each function in our code.
      \item \emph{Statistical profiling:} The code remains unchanged. At specific intervals, the profiler checks which part of the code is currently executed.
    \end{itemize}
  \end{frame}

  \begin{frame}[fragile]
    \frametitle{Profiling}

    Instrumentation:
    \begin{itemize}
      \item Pros: Easy to use, "exact" measurements
      \item Cons: Intrusive, slows down execution, can cause "heisenbugs"
    \end{itemize}

    Statistical profiling
    \begin{itemize}
      \item Pros: Not intrusive, does not slow down execution
      \item Cons: Require support from the OS (drivers)
    \end{itemize}

    \conclude{We will use instrumentation with gprof.}
  \end{frame}

  \begin{frame}[fragile]
    \frametitle{Profiling with instrumentation}

    Adding timers and counters manually is the most basic form of profiling with instrumentation.

    \begin{code}
void my_function()
{
  const double start = start_timer();

  // ...

  const double delta = stop_timer() - start;
}
    \end{code}
  \end{frame}

  \begin{frame}[fragile]
    \frametitle{Profiling with instrumentation}

    Do not use \texttt{time()}!
    \conclude{It has a low resolution of one second.}

    \begin{code}
void my_function()
{
  const time_t start = time();

  // ...

  const time_t delta = time() - start;
}
    \end{code}
  \end{frame}

  \begin{frame}[fragile]
    \frametitle{Profiling with instrumentation}

    Do not use \texttt{clock()}!
    \begin{itemize}
      \item It measures \emph{CPU time}, not \emph{elapsed time}.
      \item \texttt{CLOCKS\_PER\_SEC} is \emph{not} the number of clock ticks per second.
    \end{itemize}

    \begin{code}
void my_function()
{
  const clock_t start = clock();

  // ...

  const clock_t end = clock();
  const double delta = (end - start)/CLOCKS_PER_SEC;
}
    \end{code}
  \end{frame}

  \begin{frame}[fragile]
    \frametitle{Profiling with instrumentation}

    On Linux, use \texttt{clock\_gettime()}, which provides a high-resolution timer that measures real time.

    \begin{code}
void my_function()
{
  struct timespec start, end;
  // Use CLOCK_MONOTONIC, not CLOCK_REALTIME!
  clock_gettime(CLOCK_MONOTONIC, &start);

  // ...

  clock_gettime(CLOCK_MONOTONIC, &end);
  const double delta = time_diff(end, start);
}
    \end{code}
  \end{frame}

  \begin{frame}[fragile]
    \frametitle{Profiling with instrumentation}

    On Linux, use \texttt{clock\_gettime()}, which provides a high-resolution timer that measures real time.

    \begin{code}
double time_diff(const struct timespec *end,
                 const struct timespec *start)
{
  const double start_double
    = start->tv_sec + 1.0e-9*start->tv_nsec;
  const double end_double
    = end->tv_sec + 1.0e-9*end->tv_nsec;

  return end_double - start_double;
}
    \end{code}
  \end{frame}



  \begin{frame}[fragile]
    \frametitle{Using gprof}

    \emph{Step 1:} Compile with \texttt{-pg} flag.
    \cmd{gcc -pg -o myapp myapp.c}

    This tells the compiler to add gprof instrumentation.

    \emph{Step 2:} Run your application as usual. Doing so will produce a file named \texttt{gmon.out}.
  \end{frame}

  \begin{frame}[fragile]
    \frametitle{Using gprof}

    \emph{Step 3:} Use gprof to convert \texttt{gmon.out} to a human-readable performance profile.
    \cmd{gprof myapp gmon.out > profile.txt}
  \end{frame}
\end{document}
