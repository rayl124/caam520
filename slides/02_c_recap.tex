% Document setup
\documentclass[12pt,t]{beamer}
\usetheme[outer/progressbar=none]{metropolis}
% Algorithm environment and layout
\usepackage{algorithmicx}
\usepackage{algpseudocode}
\algloopdefx{Return}{\textbf{return} }
\usepackage{amsmath}
\usepackage{amssymb}
\usepackage{mathtools}
\usepackage[T1]{fontenc}

\usepackage{xparse}
\newbox\FBox
\NewDocumentCommand\Highlight{O{black}O{white}mO{0.5pt}O{0pt}O{0pt}}{%
    \setlength\fboxsep{#4}\sbox\FBox{\fcolorbox{#1}{#2}{#3\rule[-#5]{0pt}{#6}}}\usebox\FBox}

\usepackage{float}
\usepackage{graphicx}
\usepackage[outdir=./]{epstopdf}
\usepackage{color}
\newcommand{\red}[1]{{\color{red}#1}}
\usepackage{caption}
\definecolor{RiceBlue}{HTML}{004080}
\definecolor{BackgroundColor}{rgb}{1.0,1.0,1.0}
\setbeamercolor{frametitle}{bg=RiceBlue}
\setbeamercolor{background canvas}{bg=BackgroundColor}
\setbeamercolor{progress bar}{fg=RiceBlue}
\setbeamertemplate{caption}[default]

\usepackage{multirow}
\usepackage{tabularx}

\usepackage{xcolor}

\usepackage{listings}
\lstset{basicstyle=\footnotesize,showstringspaces=false}

\usepackage{adjustbox}

\usepackage[absolute,overlay]{textpos}

%% Overwrite font settings to make text and math font consistent.
\usepackage[sfdefault,lining]{FiraSans}
\usepackage[slantedGreek]{newtxsf}
\renewcommand*\partial{\textsf{\reflectbox{6}}}
\let\emph\relax % there's no \RedeclareTextFontCommand
\DeclareTextFontCommand{\emph}{\bfseries\em}

\renewcommand*{\vec}[1]{{\boldsymbol{#1}}}

\newcommand{\conclude}[1]{%
  \begin{itemize}
    \item[$\rightarrow$]#1
  \end{itemize}
}
\newcommand{\codeline}[2][]{%
  \begin{lstlisting}[language=c++,#1]^^J
    #2^^J
  \end{lstlisting}
}
\newcommand{\codefile}[2][]{\lstinputlisting[language=c++,frame=single,breaklines=true,#1]{#2}}
\lstnewenvironment{code}{\lstset{language=c++,frame=single}}{}


\begin{document}
  % Title page
  \title{A Recap of the C Programming Language}
  \subtitle{Computational Science II (CAAM 520)}
  \author{Christopher Thiele}
  \date{Rice University, Spring 2020}

  \setbeamertemplate{footline}{}
  \begin{frame}
    \titlepage
  \end{frame}

  \setbeamertemplate{footline}{
    \usebeamercolor[fg]{page number in head}%
    \usebeamerfont{page number in head}%
    \hspace*{\fill}\footnotesize-\insertframenumber-\hspace*{\fill}
    \vspace*{0.1in}
  }

  \begin{frame}
    \frametitle{Overview}

    Before we start using C for our purposes, we will review
    \begin{itemize}
      \item variable initialization,
      \item pointers,
      \item segmentation faults,
      \item pointers and structures,
      \item pointers to pointers, etc.,
      \item void pointers and casts,
      \item function pointers,
      \item arrays, and
      \item dynamic memory allocation.
    \end{itemize}
  \end{frame}

  \begin{frame}[fragile]
    \frametitle{Variable initialization}

    Variables must be initialized \emph{explicitly}!
    \begin{code}
int i = 123;
int k;

printf("%d\n", i);
printf("%d\n", k); // Undefined result!
    \end{code}
  \end{frame}

  \begin{frame}[fragile]
    \frametitle{Pointers}

    Pointers are variables that store the address of another variable.
    \begin{code}
int i;
int *ptr = &i;

i = 123;

printf("%p\n", ptr);  // Print the address of i.
printf("%d\n", *ptr); // Print the value of i.
    \end{code}
  \end{frame}

  \begin{frame}[fragile]
    \frametitle{Pointers}

    Variables can be modified through pointers.
    \begin{code}
int i;
int *ptr = &i;

*ptr = 123;

printf("%p\n", ptr);  // Print the address of i.
printf("%d\n", *ptr); // Print the value of i.
    \end{code}
  \end{frame}

    \begin{frame}[fragile]
    \frametitle{Pointers}

    Typical use case: output arguments

    \begin{code}
int foo()
{
  return 123;
}
    \end{code}
    vs.
    \begin{code}
void foo(int *result)
{
  *result = 123;
}
    \end{code}
  \end{frame}

  \begin{frame}[fragile]
    \frametitle{Pointers}

    Typical use case: output arguments

    Particularly useful for multiple output arguments and when using error codes!
    \begin{code}
int divide(int x, int y,
           int *quotient, int *remainder)
{
  if (y == 0) return -1;

  *quotient = x/y;
  *remainder = x%y;
  return 0;
}
    \end{code}
  \end{frame}

  \begin{frame}[fragile]
    \frametitle{Pointers}

    Like other variables, pointers must be initialized!
    \begin{code}
int *ptr;

// Undefined behavior, likely a segmentation fault!
*ptr = 123;
    \end{code}
  \end{frame}

  \begin{frame}[fragile]
    \frametitle{Pointers}

    We use \texttt{NULL} to indicate invalid pointers.
    \begin{code}
int *ptr = NULL;

// ...

// Check if pointer is valid.
if (ptr) { // Equivalent to ptr != NULL
  // ...
}

// Check if pointer is invalid.
if (!ptr) { // Equivalent to ptr == NULL
  // ...
}
    \end{code}
  \end{frame}

  \begin{frame}[fragile]
    \frametitle{Pointers}

    Pointers to structures allow more convenient notation.
    \begin{code}
typedef struct
{
  int i;
} my_struct_t;

void init(my_struct_t *s)
{
  (*s).i = 123;
  // Equivalent, but more convenient:
  s->i = 123;
}
    \end{code}
  \end{frame}

  \begin{frame}[fragile]
    \frametitle{Pointers}

    Pointers can point to variables of any type, including other pointers.
    \begin{code}
int i = 123, *ptr, **ptrptr;

ptrptr = &ptr;
*ptrptr = &i;

// Print value of ptrptr/address of ptr.
printf("%p\n", ptrptr);
// Print value of ptr/address of i.
printf("%p\n", *ptrptr);
// Print value of i.
printf("%d\n", **ptrptr);
    \end{code}
  \end{frame}

  \begin{frame}[fragile]
    \frametitle{Pointers}

    Why would we need pointers to pointers?\pause

    E.g., for functions that need have pointers as output arguments.
    \begin{code}
void ptr_max(const int *i, const int *k, int **max)
{
  if (*i > *k) {
    *max = i;
  }
  else {
    *max = k;
  }
}
    \end{code}
  \end{frame}

  \begin{frame}[fragile]
    \frametitle{Pointers}

    Constant pointers and pointers to constant things:
    \begin{code}
int i;

// What does each declaration do?
int *ptr1 = &i;
const int *ptr2 = &i;
int *const ptr3 = &i;
const int *const ptr4 = &i;
    \end{code}
  \end{frame}

  \begin{frame}[fragile]
    \frametitle{Pointers}

    Void pointers represent generic memory addresses.
    They can point to variables whose type is unknown.
    \begin{code}
int i;

void *ptr = &i;

// Error: Compiler does not know that ptr
// points to an integer!
*ptr = 123;

// We must cast void pointers before
// dereferencing them.
*((int*) ptr) = 123;
    \end{code}
  \end{frame}

  \begin{frame}[fragile]
    \frametitle{Pointers}

    Why would we need void pointers?\pause

    E.g., for functions that operate on data of any type:
    \begin{code}
// Copy any type of data.
void* memcpy(void *dst, const void *src, size_t n)

// Allocate and deallocate memory.
void* malloc(size_t size)
void free(void *ptr)
    \end{code}
  \end{frame}

  \begin{frame}[fragile]
    \frametitle{Pointers}

    Caution: C and C++ handle void pointers slightly differently!
    \begin{code}
// The following works *only* in C:
int *ptr = malloc(sizeof(int));

// In C++ we must cast explicitly.
int *ptr = (int*) malloc(sizeof(int));
    \end{code}
    \conclude{It might be a good idea to cast, as someone else could use our C code in their C++ project.}
  \end{frame}

  \begin{frame}[fragile]
    \frametitle{Function pointers}

    We can create pointers to functions as well:
    \begin{code}
int add(int i, int k) { return i + k; }

// ...

// Create a pointer to the add function.
int (*add_fptr)(int, int) = add;

// Call add through the pointer.
add_fptr(3, 7);
    \end{code}

    \conclude{Notice the parentheses in the function pointer declaration!}
  \end{frame}

  \begin{frame}[fragile]
    \frametitle{Arrays}

    C supports arrays of fixed size and any type:
    \begin{code}
int array[16];

// Indices start at zero!
for (int i = 0; i < 16; i++) {
  array[i] = i;
}

// What does this do?
array[16] = 123;
// How about this?
array[12345] = 123;
    \end{code}

    \conclude{Be very careful with indices!}
  \end{frame}

  \begin{frame}[fragile]
    \frametitle{Arrays}

    Arrays of characters (strings) are particularly common and useful.
    \begin{code}
const char string[] = "hello, world!";

printf("%d\n", strlen(string)); // 13
printf("%d\n", sizeof(string)); // 14 - why?
    \end{code}
  \end{frame}

  \begin{frame}[fragile]
    \frametitle{Dynamic memory allocation}

    We can allocate "arrays" \emph{dynamically}.
    \conclude{Technically, arrays and pointers to allocated memory are \emph{not} the same, but the differences are negligible for our purposes.}
    \begin{code}
int *array = (int*) malloc(n*sizeof(int));

if (!array) { // Equivalent to array == NULL
  fprintf(stderr, "Error: Allocation failed!\n");
  return -1;
}

// We can access the array as usual.
array[n - 1] = 123;
    \end{code}
  \end{frame}

  \begin{frame}[fragile]
    \frametitle{Dynamic memory allocation}

    Other ways to (re)allocate memory dynamically:
    \begin{code}
// Allocates *uninitialized* memory:
void* malloc(size_t size)

// Allocates memory and sets it to zero:
void* calloc(size_t num, size_t size)

// Reallocates memory, i.e., if we need more:
void* realloc(void *ptr, size_t new_size)
    \end{code}
  \end{frame}

  \begin{frame}[fragile]
    \frametitle{Dynamic memory allocation}

    If memory is allocated, it must be deallocated with \texttt{free()}.
    \begin{code}
int *array = (int*) malloc(n*sizeof(int));

// Deallocate memory.
free(array);
    \end{code}

    If memory is not released when it is no longer used, we have a \emph{memory leak}!
    \conclude{Why are leaks problematic?}
  \end{frame}

  \begin{frame}[fragile]
    \frametitle{Dynamic memory allocation}

    \emph{Example 1:} Memory is allocated repeatedly, but never deallocated.
    \begin{code}
for (int i = 0; i < 16; i++) {
  int *array = (int*) malloc(n*sizeof(int));

  // Work with array in the loop.

  // free() is missing.
}
    \end{code}
  \end{frame}

  \begin{frame}[fragile]
    \frametitle{Dynamic memory allocation}

    \emph{Example 2:} Memory is allocated, but the programmer is not aware of it.
    \begin{code}
char* get_message()
{
  char *msg = (char*) malloc(64);
  strcpy(msg, "hello, world!");
  return msg;
}
    \end{code}
  \end{frame}

  \begin{frame}[fragile]
    \frametitle{Dynamic memory allocation}

    Rule of thumb: For each call to \texttt{malloc()}, \texttt{calloc()}, or \texttt{realloc()}, there must be a matching call to \texttt{free()}.

    Note to C++ programmers:
    \begin{itemize}
      \item If memory was allocated with \texttt{new}, it \emph{must} be deallocated with \texttt{delete}.
      \item If memory was allocated with \texttt{malloc()}, \texttt{calloc()}, or \texttt{realloc()}, it \emph{must} be deallocated with \texttt{free()}.
    \end{itemize}
  \end{frame}

  \begin{frame}[fragile]
    \frametitle{Pointer arithmetic}

    C allows arithmetic with pointers:
    \begin{code}
int *array = (int*) malloc(16*sizeof(int));

// Access 7th element.
array[6] = 123;
// Equivalently:
*(array + 6) = 123;
    \end{code}

    \conclude{For \texttt{int*}, pointer arithmetic works in increments of \texttt{sizeof(int)} etc.}
  \end{frame}

  \begin{frame}[fragile]
    \frametitle{Pointer arithmetic}

    C allows arithmetic with pointers:
    \begin{code}
int *array = (int*) malloc(16*sizeof(int));

int *ptr = &array[3];
ptr += 3;
--ptr;

// Which element does this set?
*ptr = 123;
    \end{code}
  \end{frame}
\end{document}
