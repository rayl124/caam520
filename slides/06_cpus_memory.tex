% Document setup
\documentclass[12pt,t]{beamer}
\usetheme[outer/progressbar=none]{metropolis}
% Algorithm environment and layout
\usepackage{algorithmicx}
\usepackage{algpseudocode}
\algloopdefx{Return}{\textbf{return} }
\usepackage{amsmath}
\usepackage{amssymb}
\usepackage{mathtools}
\usepackage[T1]{fontenc}

\usepackage{xparse}
\newbox\FBox
\NewDocumentCommand\Highlight{O{black}O{white}mO{0.5pt}O{0pt}O{0pt}}{%
    \setlength\fboxsep{#4}\sbox\FBox{\fcolorbox{#1}{#2}{#3\rule[-#5]{0pt}{#6}}}\usebox\FBox}

\usepackage{float}
\usepackage{graphicx}
\usepackage[outdir=./]{epstopdf}
\usepackage{color}
\newcommand{\red}[1]{{\color{red}#1}}
\usepackage{caption}
\definecolor{RiceBlue}{HTML}{004080}
\definecolor{BackgroundColor}{rgb}{1.0,1.0,1.0}
\setbeamercolor{frametitle}{bg=RiceBlue}
\setbeamercolor{background canvas}{bg=BackgroundColor}
\setbeamercolor{progress bar}{fg=RiceBlue}
\setbeamertemplate{caption}[default]

\usepackage{multirow}
\usepackage{tabularx}

\usepackage{xcolor}

\usepackage{listings}
\lstset{basicstyle=\footnotesize,showstringspaces=false}

\usepackage{adjustbox}

\usepackage[absolute,overlay]{textpos}

\newcommand\blfootnote[1]{%
  \begingroup
  \renewcommand\thefootnote{}\footnote{\tiny #1}%
  \addtocounter{footnote}{-1}%
  \endgroup
}

%% Overwrite font settings to make text and math font consistent.
\usepackage[sfdefault,lining]{FiraSans}
\usepackage[slantedGreek]{newtxsf}
\renewcommand*\partial{\textsf{\reflectbox{6}}}
\let\emph\relax % there's no \RedeclareTextFontCommand
\DeclareTextFontCommand{\emph}{\bfseries\em}

\renewcommand*{\vec}[1]{{\boldsymbol{#1}}}

\newcommand{\conclude}[1]{%
  \begin{itemize}
    \item[$\rightarrow$]#1
  \end{itemize}
}
\newcommand{\codeline}[2][]{%
  \begin{lstlisting}[language=c++,#1]^^J
    #2^^J
  \end{lstlisting}
}
\newcommand{\codefile}[2][]{\lstinputlisting[language=c++,frame=single,breaklines=true,#1]{#2}}
\lstnewenvironment{code}{\lstset{language=c++,frame=single}}{}
\newcommand{\cmd}[1]{\begin{center}\texttt{#1}\end{center}}


\begin{document}
  % Title page
  \title{CPUs and Memory}
  \subtitle{Computational Science II (CAAM 520)}
  \author{Christopher Thiele}
  \date{Rice University, Spring 2020}

  \setbeamertemplate{footline}{}
  \begin{frame}
    \titlepage
  \end{frame}

  \setbeamertemplate{footline}{
    \usebeamercolor[fg]{page number in head}%
    \usebeamerfont{page number in head}%
    \hspace*{\fill}\footnotesize-\insertframenumber-\hspace*{\fill}
    \vspace*{0.1in}
  }

  \begin{frame}[fragile]
    \frametitle{Overview}

    \begin{itemize}
      \item Why do we need to know about CPU architecture and memory?
      \item What do typical CPU architectures look like?
      \item What does this mean for scientific computations?
    \end{itemize}

    \conclude{Hager \& Wellein, chapter 1}
  \end{frame}

  \begin{frame}[fragile]
    \frametitle{Computational complexity}

    Computational complexity or computational cost is often assessed by counting floating point operations (FLOP), e.g.,
    \begin{itemize}
      \item Vector dot product: $\mathcal O(n)$
      \item Matrix-vector product: $\mathcal O(n^2)$
    \end{itemize}

    While such classifications can provide useful insight, they do not always translate directly to modern computer hardware.
  \end{frame}

  \begin{frame}[fragile]
    \frametitle{CPU architecture}

    A CPU has several units for integer and floating point arithmetic.

    These units operate on (integer or floating point) registers.

    \begin{lstlisting}[frame=single]
mov eax, 1
mov ebx, 2
; Add eax and ebx, store result in eax.
add eax, ebx
    \end{lstlisting}
  \end{frame}

  \begin{frame}[fragile]
    \frametitle{CPU architecture}

    The CPU can load data from memory into registers and write from registers to memory.

    \begin{lstlisting}[frame=single]
; Load value from address given by ebx, increment,
; and write back.
mov eax, [ebx]
add eax, 1
mov [ebx], eax
    \end{lstlisting}
  \end{frame}

  \begin{frame}[fragile]
    \frametitle{CPU architecture}

    We can express the \emph{peak performance} of a CPU in floating point operations per second (FLOPS).

    For a long time, peak performance could be increased using higher and higher clock frequencies.

    \conclude{Today, physical limitations make further increases in frequency impossible or undesirable.}
  \end{frame}

  \begin{frame}[fragile]
    \frametitle{CPU architecture}

    What can be done to futher improve performance?

    \begin{itemize}
      \item Pipelining
      \item Superscalar architecture
      \item SIMD
      \item Out-of-order execution
      \item RISC instead of CISC
      \item Caches
      \item Multi-core architectures
      \item Hyperthreading
    \end{itemize}
  \end{frame}

  \begin{frame}[fragile]
    \frametitle{Pipelining}

    Instead of executing instructions one by one, break them down into simple tasks.

    Simple tasks can be performed by separate, simple units on the CPU.

    These units can be kept busy using pipelines.
  \end{frame}

  \begin{frame}[fragile]
    \frametitle{Pipelining}

    \begin{figure}
      \centering
      \includegraphics[width=0.55\linewidth]{figures/cpu_pipeline.png}
    \end{figure}
    \blfootnote{https://en.wikipedia.org/wiki/Instruction\_pipelining\#/media/File:Pipeline,\_4\_stage.svg}
  \end{frame}

  \begin{frame}[fragile]
    \frametitle{Pipelining}

    While pipelines keep CPU units busy, they also add
    \begin{itemize}
      \item latency, and
      \item the possibility of pipeline bubbles.
    \end{itemize}
  \end{frame}

  \begin{frame}[fragile]
    \frametitle{Pipelining}

    \begin{figure}
      \centering
      \includegraphics[width=0.55\linewidth]{figures/cpu_pipeline_bubble.png}
    \end{figure}
    \blfootnote{https://en.wikipedia.org/wiki/Instruction\_pipelining\#/media/File:Pipeline,\_4\_stage\_with\_bubble.svg}
  \end{frame}

  \begin{frame}[fragile]
    \frametitle{Superscalar architecture}

    A CPU can have multiple units of any given type (load, store, integer and floating point arithmetic, etc.).
    Hence, it can perform multiple operations concurrently.

    \conclude{Compare this to pipelining.}
  \end{frame}

  \begin{frame}[fragile]
    \frametitle{SIMD}

    A CPU can provide instructions that operate on multiple pieces of data at the same time.
    \conclude {Single instruction, multiple data (SIMD)}

    This is not the same as superscalar architecture!

    SIMD instructions are part of SSE, AVX, AVX2, AVX512, etc.
  \end{frame}

  \begin{frame}[fragile]
    \frametitle{Out-of-order execution}

    A CPU may reorder instructions if possible.

    \begin{lstlisting}[frame=single]
; Load value from address given by ebx
; and add it to eax.
add eax, [ebx]
; Increment ecx.
add ecx, 1
    \end{lstlisting}

    \conclude{The second instruction could be executed while the input data for the first instruction is loaded.}
  \end{frame}

  \begin{frame}[fragile]
    \frametitle{RISC instead of CISC}

    In the past, there were two types of computers/CPUs:
    \begin{itemize}
      \item Complex instruction set computers (CISC)
      \item Reduced instruction set computers (RISC)
    \end{itemize}

    While many contemporary CPUs expose a CISC interface, they are RISC CPUs internally.
  \end{frame}

  \begin{frame}[fragile]
    \frametitle{Caches}

    Modern CPUs can perform computations much faster than data can be loaded from main memory.
    \conclude{Gap between FLOPS and \emph{memory bandwidth}}

    To limit the impact of low memory bandwidth, modern CPUs have \emph{caches}, which are
    \begin{itemize}
      \item small (KBs to a few MBs) and
      \item fast (low latency, high bandwidth.)
    \end{itemize}
  \end{frame}

  \begin{frame}[fragile]
    \frametitle{Caches}

    In fact, modern CPUs typically have a hierarchy of caches:
    \begin{itemize}
      \item L1 cache: Very small, very fast
      \item L2 cache: Larger, but slower
      \item L3 cache: Even larger (a few MB), even slower
      \item Main memory: Huge, but very slow in comparison
    \end{itemize}
  \end{frame}

  \begin{frame}[fragile]
    \frametitle{Caches}

    When the CPU needs data from memory, one or more \emph{cache lines} are loaded into the cache.
    \conclude{The CPU \emph{cannot} load individual bytes from main memory.}

    Caches are used most efficiently by algorithms with
    \begin{itemize}
      \item \emph{temporal locality}, i.e., algorithms that reuse date before discarding it, or
      \item \emph{spatial locality}, i.e., algorithms that operate on contiguous blocks of data.
    \end{itemize}
  \end{frame}

  \begin{frame}[fragile]
    \frametitle{Caches}

    If the CPU requests data which is already present in a cache, we call it a \emph{cache hit}.

    Conversely, if the CPU requests data which must be loaded from main memory, we call it a \emph{cache miss}.

    \emph{Note:} CPUs also have an \emph{instruction cache}, i.e., there can be instruction cache misses, too.
  \end{frame}

  \begin{frame}[fragile]
    \frametitle{Multi-core architectures}

    The power dissipation of a CPU is proportional to the third power of the clock frequency (Hager \& Wellein).

    \conclude{Use multiple CPU \emph{cores} with lower clock frequency in a single CPU.}

    Multi-core CPUs are ubiquitous nowadays, and they allow for a variety of CPU designs.
  \end{frame}

  \begin{frame}[fragile]
    \frametitle{Hyperthreading}

    Recall that pipeline bubbles can cause CPU units, e.g., arithmetic units, to idle.

    Hyperthreading attempts to avoid pipeline bubbles by duplicating the CPUs \emph{architectural state} (registers, control, etc.).
    The CPU can then execute two applications concurrently and feed instructs from both applications into the same pipeline.

    \conclude{It depends on the (scientific) application whether hyperthreading is useful.}
  \end{frame}

  \begin{frame}[fragile]
    \frametitle{Conclusions}

    What does all of this mean in practice?

    \begin{itemize}
      \item Computational performance is a complex issue.
      \item Few applications can achieve peak performance.
      \item FLOP counting does not tell the entire story.
    \end{itemize}
  \end{frame}

  \begin{frame}[fragile]
    \frametitle{An example}

    Consider the \emph{vector triad}
    \begin{code}
for (int r = 0; r < repetitions; r++) {
  for (int i = 0; i < n; i++) {
    a[i] = b[i] + c[i]*d[i];
  }
}
    \end{code}

    Clearly, its computational complexity is $\mathcal O(n)$.
  \end{frame}

  \begin{frame}[fragile]
    \frametitle{An example}

    \begin{figure}
      \centering
      \includegraphics[width=0.8\linewidth]{figures/vec_triad_w2102.png}
    \end{figure}
  \end{frame}
\end{document}
