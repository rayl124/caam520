% Document setup
\documentclass[12pt,t]{beamer}
\usetheme[outer/progressbar=none]{metropolis}
% Algorithm environment and layout
\usepackage{algorithmicx}
\usepackage{algpseudocode}
\algloopdefx{Return}{\textbf{return} }
\usepackage{amsmath}
\usepackage{amssymb}
\usepackage{mathtools}
\usepackage[T1]{fontenc}

\usepackage{xparse}
\newbox\FBox
\NewDocumentCommand\Highlight{O{black}O{white}mO{0.5pt}O{0pt}O{0pt}}{%
    \setlength\fboxsep{#4}\sbox\FBox{\fcolorbox{#1}{#2}{#3\rule[-#5]{0pt}{#6}}}\usebox\FBox}

\usepackage{float}
\usepackage{graphicx}
\usepackage[outdir=./]{epstopdf}
\usepackage{color}
\newcommand{\red}[1]{{\color{red}#1}}
\usepackage{caption}
\definecolor{RiceBlue}{HTML}{004080}
\definecolor{BackgroundColor}{rgb}{1.0,1.0,1.0}
\setbeamercolor{frametitle}{bg=RiceBlue}
\setbeamercolor{background canvas}{bg=BackgroundColor}
\setbeamercolor{progress bar}{fg=RiceBlue}
\setbeamertemplate{caption}[default]

\usepackage{multirow}
\usepackage{tabularx}

\usepackage{xcolor}

\usepackage{listings}
\lstset{basicstyle=\footnotesize,showstringspaces=false}

\usepackage{adjustbox}

\usepackage[absolute,overlay]{textpos}

%% Overwrite font settings to make text and math font consistent.
\usepackage[sfdefault,lining]{FiraSans}
\usepackage[slantedGreek]{newtxsf}
\renewcommand*\partial{\textsf{\reflectbox{6}}}
\let\emph\relax % there's no \RedeclareTextFontCommand
\DeclareTextFontCommand{\emph}{\bfseries\em}

\renewcommand*{\vec}[1]{{\boldsymbol{#1}}}

\newcommand{\conclude}[1]{%
  \begin{itemize}
    \item[$\rightarrow$]#1
  \end{itemize}
}
\newcommand{\codeline}[2][]{%
  \begin{lstlisting}[language=c++,#1]^^J
    #2^^J
  \end{lstlisting}
}
\newcommand{\codefile}[2][]{\lstinputlisting[language=c++,frame=single,breaklines=true,#1]{#2}}
\lstnewenvironment{code}{\lstset{language=c++,frame=single}}{}
\newcommand{\cmd}[1]{\begin{center}\texttt{#1}\end{center}}


\begin{document}
  % Title page
  \title{Tools of the Trade -- Part II:\\Debugger}
  \subtitle{Computational Science II (CAAM 520)}
  \author{Christopher Thiele}
  \date{Rice University, Spring 2020}

  \setbeamertemplate{footline}{}
  \begin{frame}
    \titlepage
  \end{frame}

  \setbeamertemplate{footline}{
    \usebeamercolor[fg]{page number in head}%
    \usebeamerfont{page number in head}%
    \hspace*{\fill}\footnotesize-\insertframenumber-\hspace*{\fill}
    \vspace*{0.1in}
  }

  \begin{frame}[fragile]
    \frametitle{Debugging}

    I wrote some code, the compiler finally stopped complaining, but the code doesn't do what I want it to do.

    \conclude{Time for \emph{debugging}}
  \end{frame}

  \begin{frame}[fragile]
    \frametitle{How not to debug your code}

    Debugging with \texttt{printf()} etc. (see below) is cumbersome, inefficient and possibly misleading!
    \begin{code}
for (int i = 0; i < n; i++) {
  for (int j = i + 1; j < n; j++) {
    if (array[j] < array[i]) {
      tmp = array[i];
      array[i] = array[j];
      array[j] = tmp;
    }
  }
  printf("array[%d] = %d after iteration %d\n",
         i, array[i], i);
}
    \end{code}
  \end{frame}

  \begin{frame}[fragile]
    \frametitle{Debugging}

    Use a proper tool, i.e., a debugger like GDB!

    \emph{Preparation:} For efficient and convenient debugging, compile your code with
    \begin{itemize}
      \item debug symbols (\texttt{-g}) and
      \item possibly less optimization (e.g., \texttt{-O0} instead of \texttt{-O2}).
    \end{itemize}

    \cmd{gcc -g -O0 -o myapp myapp.c}
  \end{frame}

  \begin{frame}[fragile]
    \frametitle{Using GDB}

    To debug your program, run
    \cmd{gdb ./myapp}
    or run
    \cmd{xterm -e gdb ./myapp}
    to debug in a new terminal.
  \end{frame}

  \begin{frame}[fragile]
    \frametitle{Using GDB}

    To print your current position in the code, use the \texttt{where} and \texttt{frame} commands, e.g.,
    \cmd{(gdb) list}

    To view the surrounding source code, use the \texttt{list} command.

    To find out how you got to the current line of code, use the \texttt{backtrace} command.

    \emph{Note:} GDB allows you to use the shortest unambiguous abbreviation for any command, e.g., \texttt{l} instead of \texttt{list} etc.
  \end{frame}

  \begin{frame}[fragile]
    \frametitle{Using GDB}
    
    To run your program, use the \texttt{run} command.

    To interrupt execution, use \texttt{\^{}C} (Ctrl+C).

    To interrupt execution at a specific place, use \emph{breakpoints}:

    \texttt{(gdb) break 123\\(gdb) break my\_file.c:123\\(gdb) break my\_function\\(gdb) break 123 if my\_variable > 42}
  \end{frame}

  \begin{frame}[fragile]
    \frametitle{Using GDB}
    
    To view breakpoints, enter \texttt{info break}.

    To delete breakpoints, enter
    \begin{itemize}
      \item \texttt{delete} plus the breakpoint ID or
      \item \texttt{clear} plus the location of the breakpoint.
    \end{itemize}
  \end{frame}

  \begin{frame}[fragile]
    \frametitle{Using GDB}
    
    Watchpoints are a special type of breakpoints that interrupt execution whenever the value of a specific variable changes:

    \texttt{(gdb) watch my\_var\\(gdb) run\\Hardware watchpoint 1: my\_var\\\ \\Old value = 1\\New value = 2}
  \end{frame}

  \begin{frame}[fragile]
    \frametitle{Using GDB}
    
    To continue execution, use enter
    \begin{itemize}
      \item \texttt{continue} to continue to the next breakpoint,
      \item \texttt{step} to continue to the next line of code, entering called functions,
      \item \texttt{next} to continue to the next line of code, ignoring function calls, or
      \item \texttt{finish} to continue until the current function is left.
    \end{itemize}
  \end{frame}

  \begin{frame}[fragile]
    \frametitle{Using GDB}
    
    To print the value of a variable, use the \texttt{print} command.

    To examine a block of memory, use the \texttt{x} command, i.e.,
    \cmd{(gdb) x/nfu address}
    where
    \begin{itemize}
      \item \texttt{n} is the number of units to examine,
      \item \texttt{f} is the data format (like \texttt{printf()}, e.g., \texttt{d} for integers etc.),
      \item \texttt{u} is the unit (\texttt{b}, \texttt{h}, \texttt{w}, \texttt{g} for 1, 2, 4, 8 bytes), and
      \item \texttt{address} is the memory address.
    \end{itemize}
  \end{frame}

  \begin{frame}[fragile]
    \frametitle{Using GDB}
    
    You can modify variables using the \texttt{set var} command, e.g.,
    \cmd{set var i=123}

    To find out the type of a variable, use the \texttt{whatis} command.

    \texttt{(gdb) whatis i\\type = int\\(gdb) set var i=123}
  \end{frame}

  \begin{frame}[fragile]
    \frametitle{Using GDB}
    
    To modify memory directly, use the \texttt{set} command, e.g.,

    \texttt{(gdb) set \{int\}0x7fffffffb630=123\\(gdb) set \{char\}0x7fffffffb630='x'\\(gdb) set \{int\}my\_int\_ptr=123\\(gdb) set \{char[32]\}my\_str="hello, world!"}

    \emph{Note:} In the last two examples, it is easier to use
    \texttt{(gdb) set var *my\_int\_ptr=123\\(gdb) set var my\_str="hello, world!"}
  \end{frame}
\end{document}
