% Document setup
\documentclass[12pt,t]{beamer}
\usetheme[outer/progressbar=none]{metropolis}
% Algorithm environment and layout
\usepackage{algorithmicx}
\usepackage{algpseudocode}
\algloopdefx{Return}{\textbf{return} }
\usepackage{amsmath}
\usepackage{amssymb}
\usepackage{mathtools}
\usepackage[T1]{fontenc}

\usepackage{xparse}
\newbox\FBox
\NewDocumentCommand\Highlight{O{black}O{white}mO{0.5pt}O{0pt}O{0pt}}{%
    \setlength\fboxsep{#4}\sbox\FBox{\fcolorbox{#1}{#2}{#3\rule[-#5]{0pt}{#6}}}\usebox\FBox}

\usepackage{float}
\usepackage{graphicx}
\usepackage[outdir=./]{epstopdf}
\usepackage{color}
\newcommand{\red}[1]{{\color{red}#1}}
\usepackage{caption}
\definecolor{RiceBlue}{HTML}{004080}
\definecolor{BackgroundColor}{rgb}{1.0,1.0,1.0}
\setbeamercolor{frametitle}{bg=RiceBlue}
\setbeamercolor{background canvas}{bg=BackgroundColor}
\setbeamercolor{progress bar}{fg=RiceBlue}
\setbeamertemplate{caption}[default]

\usepackage{multirow}
\usepackage{tabularx}

\usepackage{xcolor}

\usepackage{listings}
\lstset{basicstyle=\footnotesize,showstringspaces=false}

\usepackage{adjustbox}

\usepackage[absolute,overlay]{textpos}

% Footnotes without a number
\newcommand\blfootnote[1]{%
  \begingroup
  \renewcommand\thefootnote{}\footnote{\tiny #1}%
  \addtocounter{footnote}{-1}%
  \endgroup
}

%% Overwrite font settings to make text and math font consistent.
\usepackage[sfdefault,lining]{FiraSans}
\usepackage[slantedGreek]{newtxsf}
\renewcommand*\partial{\textsf{\reflectbox{6}}}
\let\emph\relax % there's no \RedeclareTextFontCommand
\DeclareTextFontCommand{\emph}{\bfseries\em}

\renewcommand*{\vec}[1]{{\boldsymbol{#1}}}

\newcommand{\conclude}[1]{%
  \begin{itemize}
    \item[$\rightarrow$]#1
  \end{itemize}
}
\newcommand{\codeline}[2][]{%
  \begin{lstlisting}[language=c++,#1]^^J
    #2^^J
  \end{lstlisting}
}
\newcommand{\codefile}[2][]{\lstinputlisting[language=c++,frame=single,breaklines=true,#1]{#2}}
\lstnewenvironment{code}{\lstset{language=c++,frame=single}}{}
\newcommand{\cmd}[1]{\begin{center}\texttt{#1}\end{center}}


\begin{document}
  % Title page
  \title{Main Features of OpenMP}
  \subtitle{Computational Science II (CAAM 520)}
  \author{Christopher Thiele}
  \date{Rice University, Spring 2020}

  \setbeamertemplate{footline}{}
  \begin{frame}
    \titlepage
  \end{frame}

  \setbeamertemplate{footline}{
    \usebeamercolor[fg]{page number in head}%
    \usebeamerfont{page number in head}%
    \hspace*{\fill}\footnotesize-\insertframenumber-\hspace*{\fill}
    \vspace*{0.1in}
  }

  \begin{frame}[fragile]
    \frametitle{Parallel computation: an example}

    Suppose we want to integrate a function
    \[f:\left[a,b\right]\rightarrow\mathbb R\]
    numerically using the composite trapezoidal rule
    \[\int_a^b f(x)dx\approx h\sum_{i=0}^{n-1}\frac{f(x_i)+f(x_{i+1})}2,\]
    where $x_0=a$, $x_n=b$, and
    \[x_{i+1}-x_i=h\]
    for $i=0,\ldots,n-1$.
  \end{frame}

  \begin{frame}[fragile]
    \frametitle{Parallel computation: an example}

    Our goal is to implement a function \texttt{quad\_trapezoidal()} that approximates the integral \emph{in parallel} using OpenMP.

    What signature should the function have?
    \pause
    \begin{code}
double quad_trapezoidal(double (*f)(double),
                        double a,
                        double b,
                        int n);
    \end{code}
    \conclude{Let us try to implement the function with our current knowledge of OpenMP.}
  \end{frame}

  \begin{frame}[fragile]
    \frametitle{Parallel computation: an example}

    As a reasonably challenging test, we will approximate
    \[\int_0^{\frac{\pi}2}\sin(x)+x\ dx=\frac{\pi^2}8+1.\]
  \end{frame}

  \begin{frame}[fragile]
    \frametitle{Parallel computation: an example}

    Did our first attempt work?

    No, we get random results.
    \conclude{There is a \emph{data race} in the code!}

    If multiple threads modify the same variable (here: \texttt{sum}), their updates can interfere!
  \end{frame}

  \begin{frame}[fragile]
    \frametitle{Shared and private variables}

    Variables in an OpenMP application can be \emph{shared} between threads or \emph{private} to each thread.

    By default, variables are
    \begin{itemize}
      \item private if declared within a parallel region, and
      \item shared if declared before a parallel region.
    \end{itemize}

    \begin{code}
int shared_var;

#pragma omp parallel
{
  int private_var;
}
    \end{code}
  \end{frame}

  \begin{frame}[fragile]
    \frametitle{Shared and private variables}

    Alternatively, variables can be declared as shared or private for a parallel region.
    \begin{code}
int shared_var, private_var;

#pragma omp parallel shared(shared_var) \
                     private(private_var)
{
  // ...
}
    \end{code}
    \emph{Note:}
    \begin{itemize}
      \item The \texttt{shared} clause is redundant in this case.
      \item The \texttt{private} clause is necessary unless C99 is used.
    \end{itemize}
  \end{frame}

  \begin{frame}[fragile]
    \frametitle{Shared and private variables}

    \emph{Caution:} If declared outside a parallel region, the value of a private variable is \emph{undefined} inside the parallel region.
    \begin{code}
int private_var = 123;

#pragma omp parallel private(private_var)
{
  // Value of private_var is undefined!
}
    \end{code}
    \conclude{Consider using \texttt{firstprivate} instead of \texttt{private}.}
  \end{frame}

  \begin{frame}[fragile]
    \frametitle{Shared and private variables}

    \emph{Caution:} If declared outside a parallel loop, the value of a private variable is \emph{undefined} after the parallel loop.
    \begin{code}
int private_var;

#pragma omp parallel for private(private_var)
for (int i = 0; i < n; i++) {
  // ...
}
// Value of private_var is undefined!
    \end{code}
    \conclude{Consider using \texttt{lastprivate} instead of \texttt{private}.}
  \end{frame}

  \begin{frame}[fragile]
    \frametitle{Shared and private variables}

    \emph{Caution:} The code below does \emph{not} create a private array for each thread.
    \begin{code}
int *array = malloc(8*sizeof(int));

#pragma omp parallel firstprivate(array)
for (int i = 0; i < n; i++) {
  // Each thread has its own private pointer
  // to the *same* array!
}
    \end{code}
  \end{frame}


  \begin{frame}[fragile]
    \frametitle{Data races}

    Data races can occur when a shared resource, e.g., a variable, is modified.

    Data races can be hard to fix, because they can easily go unnoticed.
    \conclude{Whether the \emph{race condition} occurs is random!}
  \end{frame}

  \begin{frame}[fragile]
    \frametitle{Data races}

    \emph{Example:} Multiple threads write to a shared variable, causing a data race.
    \begin{code}
int sum = 0;

#pragma omp parallel
{
  sum += omp_get_thread_num();
}
    \end{code}
    \conclude{Note that \texttt{+=} involves both reading and writing!}
  \end{frame}

  \begin{frame}[fragile]
    \frametitle{Data races}

    To avoid data races, we must ensure that when a thread modifies a shared resource, no other thread reads from or writes to it concurrently.
    \conclude{\emph{Mutual exclusion} (mutex)}
  \end{frame}

  \begin{frame}[fragile]
    \frametitle{Locks}

    Mutexes are called \emph{locks} in OpenMP:
    \begin{code}
omp_lock_t lock;
omp_init_lock(&lock);
#pragma omp parallel
{
  // ...
  omp_set_lock(&lock);
  // Only one thread can be have the lock
  // set at any given time.
  omp_unset_lock(&lock);
  // ...
}
omp_destroy_lock(&lock);
    \end{code}
  \end{frame}

  \begin{frame}[fragile]
    \frametitle{Deadlocks}

    When locks are used, \emph{deadlocks} can occur!
    \begin{code}
void transfer(account_t a, account_t b, int amount)
{
  omp_set_lock(&a.lock);   // Lock account A.
  withdraw(a, amount);
  omp_set_lock(&b.lock);   // Lock account B.
  deposit(b, amount);
  omp_unset_lock(&b.lock); // Release account B.
  omp_unset_lock(&a.lock); // Release account A.
}
    \end{code}
    \conclude{What happens if one thread calls \texttt{transfer(a,~b,~100)} while another thread is processing \texttt{transfer(b, a, 50)}?}
  \end{frame}

  \begin{frame}[fragile]
    \frametitle{The \texttt{critical} directive}

    Locks/mutexes are cumbersome. Is there a simpler way?

    Yes, the \texttt{critical} directive:
    \begin{code}
#pragma omp parallel
{
  // ...
  #pragma omp critical
  {
    // Only one thread can be inside the
    // critical block at any given time.
  }
  // ...
}
    \end{code}
  \end{frame}

  \begin{frame}[fragile]
    \frametitle{The \texttt{for} directive}

    Distributing loop iterations among threads is cumbersome.

    Again, there is a simpler way to do it:
    \begin{code}
#pragma omp parallel
{
  #pragma omp for
  for (int i = 0; i < n; i++) {
    // The n loop iterations will be distributed
    // among threads automatically.
  }
}
    \end{code}
  \end{frame}

  \begin{frame}[fragile]
    \frametitle{The \texttt{for} directive}

    A parallel region which only contains a single for loop can be simplified as follows:

    \begin{code}
#pragma omp parallel for
for (int i = 0; i < n; i++) {
  // The n loop iterations will be distributed
  // among threads automatically.
}
    \end{code}
  \end{frame}

  \begin{frame}[fragile]
    \frametitle{The \texttt{for} directive}

    The \texttt{for} directive is not only more convenient than manual distribution of loop iterations. It is also a generalization:

    \begin{code}
#pragma omp parallel for schedule(SCHEDULE)
for (int i = 0; i < n; i++) {
  // ...
}
    \end{code}
    Possible values for \texttt{SCHEDULE} are \texttt{static}, \texttt{dynamic}, \texttt{guided}, and \texttt{auto}.
  \end{frame}

  \begin{frame}[fragile]
    \frametitle{The \texttt{for} directive}

    \emph{Caution:} Not every loop can be parallelized, as dependencies between iterations can lead to data races.

    \begin{code}
#pragma omp parallel for
for (int i = 2; i < n; i++) {
  fibonacci[i] = fibonacci[i - 1]
               + fibonacci[i - 2]; 
}
    \end{code}
    \conclude{The above code will compile without warning!}
  \end{frame}

  \begin{frame}[fragile]
    \frametitle{The \texttt{for} directive}

    Is there anything left to simplify?

    Yes, results from all iterations of a parallel loop can be combined using the \texttt{reduction} clause:
    \begin{code}
#pragma omp parallel for reduction(OP:VAR)
for (int i =  0; i < n; i++) {
  // Each thread has a private instance of VAR.
  // At the end of the loop, all values of VAR
  // are combined using the operator OP.
}
    \end{code}
    Possible values for \texttt{OP} are \texttt{+}, \texttt{-}, \texttt{*}, \texttt{min}, \texttt{max}, \texttt{\&}, \texttt{\&\&}, \texttt{|}, \texttt{||}, \texttt{\^}.
    Custom operators can also be defined.
  \end{frame}

  \begin{frame}[fragile]
    \frametitle{The \texttt{for} directive}

    \emph{Example:} Compute the Euclidean norm of a vector using reductions.
    \begin{code}
double sum;
#pragma omp parallel for reduction(+:sum)
for (int i =  0; i < n; i++) {
  sum += x[i]*x[i];
}
return sqrt(sum);
    \end{code}

    \emph{Note:} Each threads copy of the reduction variable is initialized with the neutral element of the reduction operator, e.g., zero for \texttt{+}, one for \texttt{*}, etc.
  \end{frame}

  \begin{frame}[fragile]
    \frametitle{Barriers}

    If we need to synchronize all threads, we can use a \emph{barrier}:

    No thread can get past the barrier before \emph{all} threads have reached it.
    \begin{code}
#pragma omp parallel
{
  // ...

  // Wait for other threads.
  #pragma omp barrier

  // ...
}
    \end{code}
  \end{frame}

  \begin{frame}[fragile]
    \frametitle{Barriers}

    \emph{Example:} Ensure that other threads have finished their work before we use their results.
    \begin{code}
#pragma omp parallel
{
  const int id = omp_get_thread_num();
  results[id] = do_work(id);

  #pragma omp barrier

  // Do something with results from other threads.
  do_more_work((id + 1)%omp_get_num_threads());
}
    \end{code}
  \end{frame}

  \begin{frame}[fragile]
    \frametitle{Barriers}

    The end of a parallel region or a parallel for loop is an \emph{implicit barrier}.
    \begin{code}
#pragma omp parallel
{
  #pragma omp for
  for (int i = 0; i < n; i++) {
    // ...
  } // Implicit barrier!

  // No thread gets here before all
  // threads have finished the loop.
}
    \end{code}
  \end{frame}

  \begin{frame}[fragile]
    \frametitle{Barriers}

    Barriers tend to make it easier to write correct code without data races, but they cause idling and synchronization.
    \conclude{Avoid unnecessary barriers to improve performance!}
  \end{frame}

  \begin{frame}[fragile]
    \frametitle{Barriers}

    Implicit barriers can be avoided using the \texttt{nowait} clause:
    \begin{code}
#pragma omp parallel for nowait
for (int i = 0; i < n; i++) {
  // ...
}

#pragma omp parallel for
for (int i = 0; i < n; i++) {
  // ...
}
    \end{code}
    \conclude{\emph{Caution:} This is a data race if the second loop depends on results from the first.}
  \end{frame}

  \begin{frame}[fragile]
    \frametitle{Barriers}

    If used improperly, barriers can also cause deadlocks.
    \begin{code}
#pragma omp parallel for
for (int i = 0; i < n; i++) {
  #pragma omp barrier
}
    \end{code}
    \conclude{Causes a deadlock unless each thread performs exactly the same number of loop iterations.}
  \end{frame}

  \begin{frame}[fragile]
    \frametitle{The \texttt{single} and \texttt{master} directives}

    Use the \texttt{single} directive if part of a parallel region is to be executed by only one thread.
    \begin{code}
#pragma omp parallel
{
  #pragma omp single
  {
    shared_data = malloc(size);
  }

  #pragma omp barrier

  // ...
}
    \end{code}
  \end{frame}

  \begin{frame}[fragile]
    \frametitle{The \texttt{single} and \texttt{master} directives}

    If only the master thread should execute part of a parallel region, use the \texttt{master} directive.
    \begin{code}
#pragma omp parallel
{
  #pragma omp master
  {
    // Much like "single," but the single
    // thread which executes this block
    // must be the master thread.
  }
}
    \end{code}
  \end{frame}

  \begin{frame}[fragile]
    \frametitle{The \texttt{sections} directive}

    If multiple, independent blocks of code are to be executed in parallel, use sections.
    \begin{code}
#pragma omp parallel
{
  #pragma omp sections
  {
    #pragma omp section
    {
      // 1st block
    }
    #pragma omp section
    {
      // 2nd block
...
    \end{code}
  \end{frame}
\end{document}
