% Document setup
\documentclass[12pt,t]{beamer}
\usetheme[outer/progressbar=none]{metropolis}
% Algorithm environment and layout
\usepackage{algorithmicx}
\usepackage{algpseudocode}
\algloopdefx{Return}{\textbf{return} }
\usepackage{amsmath}
\usepackage{amssymb}
\usepackage{mathtools}
\usepackage[T1]{fontenc}

\usepackage{xparse}
\newbox\FBox
\NewDocumentCommand\Highlight{O{black}O{white}mO{0.5pt}O{0pt}O{0pt}}{%
    \setlength\fboxsep{#4}\sbox\FBox{\fcolorbox{#1}{#2}{#3\rule[-#5]{0pt}{#6}}}\usebox\FBox}

\usepackage{float}
\usepackage{graphicx}
\usepackage[outdir=./]{epstopdf}
\usepackage{color}
\newcommand{\red}[1]{{\color{red}#1}}
\usepackage{caption}
\definecolor{RiceBlue}{HTML}{004080}
\definecolor{BackgroundColor}{rgb}{1.0,1.0,1.0}
\setbeamercolor{frametitle}{bg=RiceBlue}
\setbeamercolor{background canvas}{bg=BackgroundColor}
\setbeamercolor{progress bar}{fg=RiceBlue}
\setbeamertemplate{caption}[default]

\usepackage{multirow}
\usepackage{tabularx}

\usepackage{xcolor}

\usepackage{listings}
\lstset{basicstyle=\footnotesize,showstringspaces=false}

\usepackage{adjustbox}

\usepackage[absolute,overlay]{textpos}

% Footnotes without a number
\newcommand\blfootnote[1]{%
  \begingroup
  \renewcommand\thefootnote{}\footnote{\tiny #1}%
  \addtocounter{footnote}{-1}%
  \endgroup
}

%% Overwrite font settings to make text and math font consistent.
\usepackage[sfdefault,lining]{FiraSans}
\usepackage[slantedGreek]{newtxsf}
\renewcommand*\partial{\textsf{\reflectbox{6}}}
\let\emph\relax % there's no \RedeclareTextFontCommand
\DeclareTextFontCommand{\emph}{\bfseries\em}

\renewcommand*{\vec}[1]{{\boldsymbol{#1}}}

\newcommand{\conclude}[1]{%
  \begin{itemize}
    \item[$\rightarrow$]#1
  \end{itemize}
}
\newcommand{\codeline}[2][]{%
  \begin{lstlisting}[language=c++,#1]^^J
    #2^^J
  \end{lstlisting}
}
\newcommand{\codefile}[2][]{\lstinputlisting[language=c++,frame=single,breaklines=true,#1]{#2}}
\lstnewenvironment{code}{\lstset{language=c++,frame=single}}{}
\newcommand{\cmd}[1]{\begin{center}\texttt{#1}\end{center}}


\begin{document}
  % Title page
  \title{OpenMP: An Advanced Example}
  \subtitle{Computational Science II (CAAM 520)}
  \author{Christopher Thiele}
  \date{Rice University, Spring 2020}

  \setbeamertemplate{footline}{}
  \begin{frame}
    \titlepage
  \end{frame}

  \setbeamertemplate{footline}{
    \usebeamercolor[fg]{page number in head}%
    \usebeamerfont{page number in head}%
    \hspace*{\fill}\footnotesize-\insertframenumber-\hspace*{\fill}
    \vspace*{0.1in}
  }

  \begin{frame}[fragile]
    \frametitle{Motivation}

    Our examples so far were simple in the sense that adding an OpenMP directive to a loop was usually sufficient.

    In general, parallelization can be more complicated due to dependencies between loop iterations.
    \conclude{Let us consider such an example.}
  \end{frame}

  \begin{frame}[fragile]
    \frametitle{The heat equation}

    We want to solve the heat equation
    \begin{align*}
      \partial_t u-\Delta u &= f && \text{in }\Omega\times (0,T),\\
      u &= 0 && \text{on }\partial\Omega\times (0, T),\\
      u &= u_0 && \text{on }\Omega\times\left\{0\right\},
    \end{align*}
    where $\Omega=\left[0,1\right]^3$, $u$ is the temperature, and $f$ describes heat sources and heat sinks inside $\Omega$.
  \end{frame}

  \begin{frame}[fragile]
    \frametitle{Poisson's equation}

    Instead of solving the full time-dependent problem, we are interested in the \emph{steady state} solution which satisfies $\partial_t u=0$.

    This leads to the \emph{Poisson problem}
    \begin{align*}
      -\Delta u &= f && \text{in }\Omega,\\
      u &= 0 && \text{on }\partial\Omega.
    \end{align*}
  \end{frame}

  \begin{frame}[fragile]
    \frametitle{Finite difference discretization}

    To discretize the equation, we introduce $n^3$ grid points
    \[
      x_{ijk}=\left(ih,jh,kh\right)^T,
    \]
    where $i,j,k=0,\ldots,n-1$ and $h=\frac 1{n-1}$.

    For convenience, we define
    \[
      u_{ijk}=u(x_{ijk})
    \]
    etc.
  \end{frame}

  \begin{frame}[fragile]
    \frametitle{Finite difference discretization}

    Discretizing the equation using finite differences (at an interior point $x_{ijk}$) yields
    \[
      \frac{-u_{i-1jk} - u_{ij-1k} - u_{ijk-1} + 6u_{ijk} - u_{i+1jk} - u_{ij+1k} - u_{ijk+1}}{h^2}=f_{ijk},
    \]
    or equivalently
    \[
      u_{ijk}=\frac{h^2f_{ijk} + u_{i-1jk} + u_{ij-1k} + u_{ijk-1} + u_{i+1jk} + u_{ij+1k} + u_{ijk+1}}6,
    \]
    i.e., a linear system.
  \end{frame}

  \begin{frame}[fragile]
    \frametitle{The Jacobi iteration}

    The linear system can be written in matrix form and solved, e.g., with Gaussian elimination.

    Since it is \emph{sparse}, it can also be solved iteratively.
    The simplest iterative method is the \emph{Jacobi} iteration
    \[
      u_{ijk}^\text{new}\leftarrow\frac{h^2f_{ijk} + u_{i-1jk}^\text{old} + u_{ij-1k}^\text{old} + u_{ijk-1}^\text{old} + u_{i+1jk}^\text{old} + u_{ij+1k}^\text{old} + u_{ijk+1}^\text{old}}6,
    \]
    \conclude{This method is straightforward to parallelize, but requires many iterations.}
  \end{frame}

  \begin{frame}[fragile]
    \frametitle{The Gauss--Seidel iteration}

    Another method, which converges faster in the sense that it requires fewer iterations, is the \emph{Gauss--Seidel} iteration
    \[
      u_{ijk}^\text{new}\leftarrow\frac{h^2f_{ijk} + u_{i-1jk}^\text{new} + u_{ij-1k}^\text{new} + u_{ijk-1}^\text{new} + u_{i+1jk}^\text{old} + u_{ij+1k}^\text{old} + u_{ijk+1}^\text{old}}6.
    \]
    \conclude{Since $u_{ijk}^\text{new}$ depends on updated values at other grid points, how can we parallelize the iteration?}
  \end{frame}

  \begin{frame}[fragile]
    \frametitle{Remark}

    In practice, the tradeoff between the number of iterations and the per-iteration cost is nontrivial.

    It depends on the problem at hand whether the Jacobi method or the Gauss--Seidel method yields in lower time-to-solution.
  \end{frame}
\end{document}
