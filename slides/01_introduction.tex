% Document setup
\documentclass[12pt,t]{beamer}
\usetheme[outer/progressbar=none]{metropolis}
% Algorithm environment and layout
\usepackage{algorithmicx}
\usepackage{algpseudocode}
\algloopdefx{Return}{\textbf{return} }
\usepackage{amsmath}
\usepackage{amssymb}
\usepackage{mathtools}
\usepackage[T1]{fontenc}

\usepackage{xparse}
\newbox\FBox
\NewDocumentCommand\Highlight{O{black}O{white}mO{0.5pt}O{0pt}O{0pt}}{%
    \setlength\fboxsep{#4}\sbox\FBox{\fcolorbox{#1}{#2}{#3\rule[-#5]{0pt}{#6}}}\usebox\FBox}

\usepackage{float}
\usepackage{graphicx}
\usepackage[outdir=./]{epstopdf}
\usepackage{color}
\newcommand{\red}[1]{{\color{red}#1}}
\usepackage{caption}
\definecolor{RiceBlue}{HTML}{004080}
\definecolor{BackgroundColor}{rgb}{1.0,1.0,1.0}
\setbeamercolor{frametitle}{bg=RiceBlue}
\setbeamercolor{background canvas}{bg=BackgroundColor}
\setbeamercolor{progress bar}{fg=RiceBlue}
\setbeamertemplate{caption}[default]

\usepackage{multirow}
\usepackage{tabularx}

\usepackage{xcolor}

\usepackage{listings}
\lstset{basicstyle=\footnotesize, xleftmargin=-0.25in}

\hypersetup{colorlinks=true}

\usepackage{adjustbox}

\usepackage[absolute,overlay]{textpos}

%% Overwrite font settings to make text and math font consistent.
\usepackage[sfdefault,lining]{FiraSans}
\usepackage[slantedGreek]{newtxsf}
\renewcommand*\partial{\textsf{\reflectbox{6}}}
\let\emph\relax % there's no \RedeclareTextFontCommand
\DeclareTextFontCommand{\emph}{\bfseries\em}

\renewcommand*{\vec}[1]{{\boldsymbol{#1}}}

\begin{document}
  % Title page
  \title{Introduction}
  \subtitle{Computational Science II (CAAM 520)}
  \author{Christopher Thiele}
  \date{Rice University, Spring 2020}

  \setbeamertemplate{footline}{}
  \begin{frame}
    \titlepage
  \end{frame}

  \setbeamertemplate{footline}{
    \usebeamercolor[fg]{page number in head}%
    \usebeamerfont{page number in head}%
    \hspace*{\fill}\footnotesize-\insertframenumber-\hspace*{\fill}
    \vspace*{0.1in}
  }

  \begin{frame}
    \frametitle{General information}

    \begin{tabular}{ll}
      Course: & Computational Science II (CAAM 520)\\
      Class: & MWF 1:00pm -- 1:50pm in DCH 1075\\
      Instructor: & Christopher Thiele\\
      Office: & DCH 3015\\
      Email: & ct37@rice.edu\\
      Office hours: & W 2:00pm -- 3:00pm in DCH 3110\\
    \end{tabular}
    \begin{itemize}
      \item[$\rightarrow$] All information on class website: \href{https://cthl.github.io/caam520}{https://cthl.github.io/caam520}
    \end{itemize}
  \end{frame}

  \begin{frame}
    \frametitle{Motivation}

    Today's computers have multi-core CPUs that cannot be utilized by \emph{sequential}, i.e., non-parallel software.
    \begin{itemize}
      \item This is true even for laptop computers, phones, etc.
      \item Tools and paradigms are needed to use such hardware.
      \item We will learn about \emph{shared memory parallelism}\\$\rightarrow$ OpenMP
    \end{itemize}
  \end{frame}

  \begin{frame}
    \frametitle{Motivation}

    Multiple computers will always be faster than a single computer.
    \begin{itemize}
      \item Large scientific applications require computer \emph{clusters}.
      \item Requires \emph{additional} tools and paradigms.
      \item We will learn about \emph{distributed memory parallelism}\\$\rightarrow$ MPI (Message Passing Interface)
    \end{itemize}
  \end{frame}

  \begin{frame}
    \frametitle{Motivation}

    Even though parallelism makes programming more difficult, we must develop software that is
    \begin{itemize}
      \item correct,
      \item robust,
      \item maintainable,
      \item extensible, and
      \item portable.
    \end{itemize}

    \begin{itemize}
      \item[$\rightarrow$] We will touch on aspects of software engineering where appropriate.
    \end{itemize}
  \end{frame}

  \begin{frame}
    \frametitle{Overview}

    In this course, we will learn to
    \begin{itemize}
      \item understand the benefits and challenges of parallel computing on current multi-core computers,
      \item design parallel algorithms for scientific applications,
      \item implement parallel algorithms with OpenMP, MPI, and hybrid approaches,
      \item assess the suitability of parallel programming techniques for scientific applications,
      \item measure the performance and scalability of scientific software, and
      \item develop robust, extensible, and maintainable scientific code.
    \end{itemize}
  \end{frame}

  \begin{frame}
    \frametitle{Tentative list of topics}

    \begin{itemize}
      \item Recap of C (specifically pointers, memory management)
      \item Standard development tools (GCC, make, GDB, gprof, Valgrind, Git)
      \item CPU and memory architecture on current multi-core systems
      \item Parallelism -- some theory
      \item Introduction to shared memory parallelism with OpenMP
      \item Introduction to distributed memory parallelism with MPI
      \item Hybrid parallelism
    \end{itemize}
  \end{frame}

  \begin{frame}
    \frametitle{Tentative list of topics (optional)}

    \begin{itemize}
      \item Advanced MPI (non-blocking and one-sided communication)
      \item Scientific software at scale (PETSc)
    \end{itemize}
  \end{frame}

  \begin{frame}
    \frametitle{Texts and materials}

    There are no required texts for this course.

    The following texts and manuals are recommended:
    \begin{itemize}
      \item Introduction to High Performance Computing for Scientists and Engineers by Hager and Wellein (2011)
      \item The C Programming Language (2nd ed.) by Kernighan and Ritchie (1988)
    \end{itemize}
    \begin{itemize}
      \item[$\rightarrow$] See website for additional material.
    \end{itemize}
  \end{frame}

  \begin{frame}
    \frametitle{Prerequisites}

    \begin{itemize}
      \item You should be familiar with C and the Linux/Unix command line.
      \item You should have basic knowledge of Git.
      \item You must have access to a computer on which you can install software, preferably a laptop computer.
      \item You must have (or create) a GitHub account.
    \end{itemize}
  \end{frame}

  \begin{frame}
    \frametitle{Grading}

    \begin{itemize}
      \item Grades will be based on homework submissions.
      \item Some homework problems might be pledged.
      \item There will not be any exams.
    \end{itemize}
  \end{frame}

  \begin{frame}
    \frametitle{Homework}

    Homework will involve programming and writing a report.
    \begin{itemize}
      \item Anticipate a homework every 2 -- 3 weeks (one per section).
      \item Submit code and reports to a private GitHub repository (will be provided).
      \item Late submissions will incur a penalty of 10\% per day. Ask \emph{in advance} for exceptions.
    \end{itemize}
  \end{frame}

  \begin{frame}
    \frametitle{Homework}

    Reports must be submitted as PDF files.

    Code can be written in C or C++ (any standard) as long as
    \begin{itemize}
      \item a Makefile is provided, and
      \item interfaces are not violated.
    \end{itemize}
  \end{frame}

  \begin{frame}[fragile]
    \frametitle{Homework}

    \emph{Example 1:} The task is to implement a function
    \begin{lstlisting}[language=c]
      int norm2(const double *x, int n)
    \end{lstlisting}

    You can implement this function in C or C++, but you must not change its name or signature, i.e., do not implement
    \begin{lstlisting}[language=c++]
      double euclidean_norm(const std::vector<double> &x)
    \end{lstlisting}
  \end{frame}

  \begin{frame}[fragile]
    \frametitle{Homework}

    \emph{Example 2:} The task is to implement a structure
    \begin{lstlisting}[language=c]
      typedef struct
      {
        double *x;
        int n;
      } my_vector;
    \end{lstlisting}

    You can implement this function in C or C++, but do not implement
    \begin{lstlisting}[language=c++]
      class my_vector
      {
      private:
        double *x;
        int n;
      };
    \end{lstlisting}
  \end{frame}

  \begin{frame}
    \frametitle{Homework}

    A Linux-based virtual machine (VM) with all software required for the homework is available \href{https://rice.box.com/s/1fsjlvmved24ffkwujcw5m450eovpyqq}{here}.

    It can be used with VirtualBox (\href{https://www.virtualbox.org/}{https://www.virtualbox.org/}) on Linux, macOS, and Windows.

    Use of the VM is strongly encouraged.
    \begin{itemize}
      \item Some software is difficult to install on some systems, e.g., GCC and GDB on macOS.
      \item Using the VM ensures that your code works on the grader's system.
    \end{itemize}
  \end{frame}
\end{document}
